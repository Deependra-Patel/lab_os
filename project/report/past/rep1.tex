\documentclass[11pt]{article}

\usepackage[utf8]{inputenc}
\usepackage{amsmath}
\usepackage{listings}
\usepackage{amsfonts}
\usepackage{amssymb}
\usepackage{color}
\usepackage{inputenc}
\usepackage{graphicx}

\usepackage[left=1.2in,top=1.2in,right=1.2in,bottom=1.0in]{geometry} % Document margins

\title{\textbf{Operating System Lab Project \\ 
	Theme B \\
    Virtual Memory for the Geek OS \\
    Report for this week
	 }}

\definecolor{mygreen}{rgb}{0,0.6,0}
\definecolor{mygray}{rgb}{0.5,0.5,0.5}
\definecolor{mymauve}{rgb}{0.58,0,0.82}

\author{Rohit Kumar \\ 120050028 \\ rohit@cse.iitb.ac.in
		\and 
		Deependra Patel \\ 120050032 \\ deependra@cse.iitb.ac.in
		\and
		Jayesh Bageriya \\ 120050022 \\ jayeshbageriya@cse.iitb.ac.in
        \\
		\and 
		Abhishek Thakur \\ 120050008 \\ abhishekthakur@cse.iitb.ac.in
		}

\date{\today}

\lstset{ %
  backgroundcolor=\color{white},   % choose the background color; you must add \usepackage{color} or \usepackage{xcolor}
  basicstyle=\footnotesize,        % the size of the fonts that are used for the code
  breakatwhitespace=false,         % sets if automatic breaks should only happen at whitespace
  breaklines=true,                 % sets automatic line breaking
  captionpos=b,                    % sets the caption-position to bottom
  commentstyle=\color{mygreen},    % comment style
  escapeinside={\%*}{*)},          % if you want to add LaTeX within your code
  extendedchars=true,              % lets you use non-ASCII characters; for 8-bits encodings only, does not work with UTF-8
  frame=single,                    % adds a frame around the code
  keepspaces=true,                 % keeps spaces in text, useful for keeping indentation of code (possibly needs columns=flexible)
  keywordstyle=\color{red},       % keyword style
  language=vhdl,                 % the language of the code
  numbers=left,                    % where to put the line-numbers; possible values are (none, left, right)
  numbersep=5pt,                   % how far the line-numbers are from the code
  numberstyle=\tiny\color{mygray}, % the style that is used for the line-numbers
  rulecolor=\color{black},         % if not set, the frame-color may be changed on line-breaks within not-black text (e.g. comments (green here))
  showspaces=false,                % show spaces everywhere adding particular underscores; it overrides 'showstringspaces'
  showstringspaces=false,          % underline spaces within strings only
  showtabs=false,                  % show tabs within strings adding particular underscores
  stepnumber=2,                    % the step between two line-numbers. If it's 1, each line will be numbered
  stringstyle=\color{mymauve},     % string literal style
  tabsize=2,                       % sets default tabsize to 2 spaces
  title=\lstname                   % show the filename of files included with \lstinputlisting; also try caption instead of title
}

\begin{document}

\maketitle



\section{Modification for using page table (Multilevel Page Table)}
\paragraph{}

The first step is to modify the code to use page tables rather than just segments to
provide memory protection. Now every region of the memory that is being accessed should have an entry in the page table.

Now, there will be a single page table for all kernel only threads, and a page table for each user process. In addition, the page tables for user mode processes will also contain entries to address the kernel mode memory (memory that can be accessed only by kernel). The kernel memory should be a one to one mapping of all of the physical memory in the processor. The page
table entries for this memory have to be marked so that this memory is only accessible from kernel mode.

This is achieved by creating a page directory and page table entries for the kernel threads by writing code which will 
initialize the page tables and enables paging mode in the processor.

\begin{lstlisting}[language=c]

/*
 * Kernel Page Directory initialization
 */
pde_t *g_kernel_pde = {0};

\end{lstlisting}


Now, the page tables are set. A new page directory is allocated using Alloc\_Page() and then
page tables for the entire region that will be mapped into this memory context are allocated (done in the for loop in below code). Appropriate fields in the page tables and page directories, (represented by the pde\_t and pde\_t data-types) are filled.

\newpage

\begin{lstlisting}[language=c]

/*
 * Page directory entry datatype.
 * If marked as present, it specifies the physical address
 * and permissions of a page table.
 */
typedef struct {
    uint_t present:1;
    uint_t flags:4;
    uint_t accessed:1;
    uint_t reserved:1;
    uint_t largePages:1;
    uint_t globalPage:1;
    uint_t kernelInfo:3;
    uint_t pageTableBaseAddr:20;
} pde_t;

/*
 * Page table entry datatype.
 * If marked as present, it specifies the physical address
 * and permissions of a page of memory.
 */
typedef struct {
    uint_t present:1;
    uint_t flags:4;
    uint_t accessed:1;
    uint_t dirty:1;
    uint_t pteAttribute:1;
    uint_t globalPage:1;
    uint_t kernelInfo:3;
    uint_t pageBaseAddr:20;
} pte_t;


\end{lstlisting}

% add c code

\begin{lstlisting}[language=c]


/*
 * Initialize virtual memory by building page tables
 * for the kernel and physical memory.
 */
pde_t *g_kernel_pde = {0};

void Init_VM(struct Boot_Info *bootInfo) {
    int  kernel_pde_entries;
    int  whole_pages;
    int  i,j;
    uint_t mem_addr;
    pte_t * cur_pte;
    whole_pages = bootInfo->memSizeKB/4;
    Print("whole pages are %d\n",whole_pages);
    kernel_pde_entries = whole_pages/NUM_PAGE_DIR_ENTRIES + (whole_pages % NUM_PAGE_DIR_ENTRIES == 0 ? 0:1);
    Print("kdep %d\n",kernel_pde_entries);
    Print("the kernel_pde_entries is %d\n",kernel_pde_entries);
    //KASSERT(0);

    g_kernel_pde=(pde_t *)Alloc_Page();
    Print("KERNALADD: %d\n", (int)g_kernel_pde);

    memset(g_kernel_pde,'\0',PAGE_SIZE);

    pde_t * cur_pde_entry;
    cur_pde_entry=g_kernel_pde;
    mem_addr=0;
    for(i=0;i<kernel_pde_entries;i++){
        cur_pde_entry->present=1;
        cur_pde_entry->flags=VM_WRITE | VM_USER;

        Print("I: %d\n", i);

        cur_pte=(pte_t *)Alloc_Page();
        KASSERT(cur_pte!=NULL);

        memset(cur_pte,'\0',PAGE_SIZE);
        cur_pde_entry->pageTableBaseAddr=(uint_t)cur_pte>>12;
        mem_addr = i*NUM_PAGE_TABLE_ENTRIES*PAGE_SIZE;
        int last_pagetable_num;
        last_pagetable_num=whole_pages%NUM_PAGE_TABLE_ENTRIES;
        if(last_pagetable_num==0 || i!=(kernel_pde_entries-1)){
          last_pagetable_num=NUM_PAGE_TABLE_ENTRIES;
        }

        for(j=0;j<last_pagetable_num;j++){
            cur_pte->present=1;
            cur_pte->flags=VM_WRITE|VM_USER;
            cur_pte->pageBaseAddr=mem_addr>>12;

            cur_pte++;
            mem_addr+=PAGE_SIZE;
        }
        cur_pde_entry++;
    }
    Enable_Paging(g_kernel_pde);
    Install_Interrupt_Handler(14,Page_Fault_Handler);
}
    
\end{lstlisting}


Finally, to enable paging for the first time, an assembly routine, Enable Paging() is called that takes the base address of the page directory as a parameter and then load the passed page directory address into register cr3, and then set the paging bit in cr0.

\begin{lstlisting}[language=c]

;
; Start paging
;	load crt3 with the passed page directory pointer
;	enable paging bit in cr2
align 8
Enable_Paging:
	mov	eax, [esp+4]
	mov	cr3, eax
	mov	eax, cr3
	mov	cr3, eax
	mov	ebx, cr0
	or	ebx, 0x80000000
	mov	cr0, ebx
	ret
    
\end{lstlisting}

\section{Handling Page Faults}
\paragraph{}


For working with paging, the operating system needs to handle page faults. To do this a page fault interrupt handler should be present.

The first thing the page fault handler will need to do is to determine the address of the page fault which can be found out  by calling the Get Page Fault Address() function (present in  geekos/paging.h. Also, the errorCode field of the Interrupt
State data structure passed to the page fault interrupt handler contains information about the faulting
access. This information is defined in the faultcode\_t data type (defined in geekos/paging.h).

\begin{lstlisting}[language=c]
/*
 * Datatype representing the hardware error code
 * pushed onto the stack by the processor on a page fault.
 * The error code is stored in the "errorCode" field
 * of the Interrupt_State struct.
 */
typedef struct {
    uint_t protectionViolation:1;
    uint_t writeFault:1;
    uint_t userModeFault:1;
    uint_t reservedBitFault:1;
    uint_t reserved:28;
} faultcode_t;

\end{lstlisting}



Given a virtual address at which page fault arises, we get the index of the page directory corresponding to that address (10 MSB's) and the index of the page level of  that page directory (next 10 MSB's). The address of the page directory entry is found by adding the base pointer of the first level page table (directory level) (userContext->pageDir) and the page directoy index obtained as above. This consists of 20 bits.

If the directory entry is present in the directory level page table, then the exact address of the page entry is obtained by adding page\_addr of the virtual address(obtained above) to left shifted address of page directory entry(obtained above).

\pagebreak

\begin{lstlisting}[language=c]

/*
 * Handler for page faults.
 * You should call the Install_Interrupt_Handler() function to
 * register this function as the handler for interrupt 14.
 */
/*static*/ void Page_Fault_Handler(struct Interrupt_State *state) {
    ulong_t address;
    faultcode_t faultCode;

    KASSERT(!Interrupts_Enabled());

    /* Get the address that caused the page fault */
    address = Get_Page_Fault_Address();
    Debug("Page fault @%lx\n", address);

    if (address < 0xfec01000 && address > 0xf0000000) {
        KASSERT0(0, "page fault address in APIC/IOAPIC range\n");
    }

    /* Get the fault code */
    faultCode = *((faultcode_t *) & (state->errorCode));

    /* rest of your handling code here */
   // TODO_P(PROJECT_VIRTUAL_MEMORY_B, "handle page faults");

    ulong_t page_dir_addr= PAGE_DIRECTORY_INDEX(address);
    ulong_t page_addr= PAGE_TABLE_INDEX(address);
    pde_t * page_dir_entry=(pde_t*)userContext->pageDir+page_dir_addr;
    pte_t * page_entry= NULL;
    
    if(page_dir_entry->present)
    {
        page_entry=(pte_t*)((page_dir_entry->pageTableBaseAddr) << 12);
        page_entry+=page_addr;
    }
    else
    {
		// Error occured. Not implemented for this part
    }
\end{lstlisting}


\section{Things left to be done}
\paragraph{}

\begin{itemize}
\item Finish Page Fault Handler and Page Replacement
\item Implement Pinned Pages
\item Creating swap space for a process in a file
\item Complete user level virtual memory implementation
\item TLB
\end{itemize}

\section{Work Distribution}
\paragraph{}
\begin{itemize}
\item Deependra and Rohit : Multilevel Page Table Implementation (Init\_VM)
\item Jayesh and Abhishek : Page Fault Handler
\end{itemize}

\end{document}
